%
% Einige n�tzliche Makros f�r die Verwendung in Texten
%------------------------------------------------------
%
% Autor:   Dirk Benyoucef
% Version: V1.6
% Letzte �nderung : 55.02.2008
%

% Classe KOMA
%\renewcommand*{\capfont}{\captionfont}
%\renewcommand*{\caplabelfont}{\captionlabelfont}
\renewcommand{\figurename}{Bild}
\renewcommand{\tablename}{Tabelle}


% Umdefinition der Symbole f�r item
\renewcommand{\labelitemiii}{\FilledSmallSquare}
\renewcommand{\labelitemi}{\FilledSmallTriangleRight}
\renewcommand{\labelitemii}{\FilledSmallDiamondshape}

% Laborname
\newcommand{\DCSP}{\emph{Digital Communications \& Signal Processing }}

% Boxen definieren
\theoremheaderfont{\normalfont\bfseries}
\theoremstyle{break}
\theorembodyfont{\slshape}

\newtheorem{satzenv}{Anforderungen}[chapter]

\definecolor{boxgray}{gray}{0.9}

\newenvironment{graubox}{%
  \setlength{\fboxsep}{10pt}%
  \def\FrameCommand{\fcolorbox{black}{boxgray}}%
  \MakeFramed {\advance\hsize-30pt \FrameRestore}}%
 {\endMakeFramed}

\newenvironment{inhalt}%
{%
  \begin{graubox}
  \begin{satzenv}
}{%
  \end{satzenv}
  \end{graubox}
}

\newenvironment{grau_verbatim}%
{%
  \begin{graubox}
  \begin{verbatim}
}{%
  \end{verbatim}
  \end{graubox}
}


% Real- Imagin�rteil
\newcommand{\real}[1]{\operatorname{Re}{\left\{ #1 \right\}}}
\newcommand{\imag}[1]{\operatorname{Im}{\left\{ #1 \right\}}}

% Spezielle Befehle f�r OFDM
\newcommand{\Comment}[1]{}
\newcommand{\complex}[1]{\underline{#1}}
\newcommand{\dft}[1]{\operatorname{DFT}\left\{ #1 \right\}}
\newcommand{\idft}[1]{\operatorname{IDFT}\left\{ #1 \right\}}

\newcommand{\domega}{e^{j 2 \pi f T}}
\newcommand{\aOmega}{e^{j \Omega}}
\newcommand{\vOmega}{\Omega}

%
% STANDARDISIERUNG einiger SYMBOLE
%

% Mengenbuchstaben
%
\newcommand{\mm}[1]{\ensuremath{\mathbb{#1}}}
\newcommand{\mc}{\mm{C}}            % komplexe Zahlen
\newcommand{\mn}{\mm{N}}            % nat�rliche Zahlen
\newcommand{\mg}{\mm{G}}
\newcommand{\mi}{\mm{I}}
\newcommand{\mq}{\mm{Q}}
\newcommand{\mr}{\mm{R}}            % reelle Zahlen
\newcommand{\mw}{\mm{W}}
\newcommand{\mz}{\mm{Z}}            % ganze Zahlen
\newcommand{\rz}{{\mathbb R}}       % reelle Zahlen
\newcommand{\cz}{{\mathbb C}}       % komplexe Zahlen
\newcommand{\zz}{{\mathbb Z}}       % ganze Zahlen
\newcommand{\nz}{{\mathbb N}}       % nat�rliche Zahlen
\newcommand{\Nnull}{{\mathbb N}_0}  % nat�rliche Zahlen ohne die Null

\newcommand{\ma}{\mathcal{A}}            % Menge von A
\newcommand{\mb}{\mathcal{B}}            % Menge von B
\newcommand{\mt}{\mathcal{T}}
\newcommand{\mx}{\mathcal{X}}

% Einige Normen
\newcommand{\norm} [1]{\left\| #1 \right\|}
\newcommand{\lnorm}[1]{\left\| #1 \right\|_{l^2}}
\newcommand{\Lnorm}[1]{\left\| #1 \right\|_{L^2}}
\newcommand{\wnorm}[1]{\left\| #1 \right\|_{\mathrm W}}

% Skalarprodukte
\newcommand{\skalar} [2]{\left\langle #1, #2\right\rangle}
\newcommand{\lskalar}[2]{\left\langle #1, #2\right\rangle_{l^2}}
\newcommand{\Lskalar}[2]{\left\langle #1, #2\right\rangle_{L^2}}

% Erwartungswert
\newcommand{\erw}[1]{\mathcal{E}{\left\{#1\right\}}}
\newcommand{\var}[1]{\operatorname{var}\left\{#1\right\}}

% Vektoralebra
\newcommand{\rot}{\operatorname{rot}}
\newcommand{\divergenz}{\operatorname{div}}
\newcommand{\grad}{\operatorname{grad}}

% Neue Signale oder Operatoren
\newcommand{\dif}{\operatorname{d \!}}
\newcommand{\rect}{\operatorname{rect}}
\newcommand{\erfc}{\operatorname{erfc}}
\newcommand{\erf}{\operatorname{erf}}
\newcommand{\si}{\operatorname{si}}
\newcommand{\Si}{\operatorname{Si}}
\newcommand{\adj}{\operatorname{adj}}
\newcommand{\sign}[1]{\operatorname{sign}\left\{ #1 \right\}}
\newcommand{\spur}[1]{\operatorname{spur}\left\{ #1 \right\}}
\newcommand{\diag}[1]{\operatorname{diag}\left\{ #1 \right\}}
\newcommand{\Per}{\operatorname{Per}}
\newcommand{\Min}{\operatorname{Min}}


% Transformationen
\newcommand{\trafo}[1]{\mathcal{T}{\left\{#1\right\}}}
\newcommand{\four}[1]{\mathcal{F}{\left\{#1\right\}}}
\newcommand{\fourdft}[1]{\mathcal{F}_{DFT}{\left\{#1\right\}}}
\newcommand{\invfour}[1]{\mathcal{F}^{-1}{\left\{#1\right\}}}
\newcommand{\invfourdft}[1]{\mathcal{F}^{-1}_{DFT}{\left\{#1\right\}}}
\newcommand{\stft}[1]{\mathcal{F}{_{STFT}\left\{#1\right\}}}
\newcommand{\ztrafo}[1]{\mathcal{Z}{\left\{#1\right\}}}
\newcommand{\invztrafo}[1]{\mathcal{Z}^{-1}{\left\{#1\right\}}}
\newcommand{\hilbert}[1]{\mathcal{H}\left\{ #1 \right\}}
\newcommand{\hilbertinv}[1]{\mathcal{H}^{-1}\left\{#1 \right\}}
\newcommand{\stftmert}[1]{\mathcal{F}_{#1}^\gamma(\omega,\tau)}
\newcommand{\fourmert}[1]{\mathcal{F}_{#1}(\omega,\tau)}
\newcommand{\wave}[1]{\mathcal{W}{\left\{#1\right\}}}
\newcommand{\wavep}[1]{\mathcal{W}{_{WP}\left\{#1\right\}}}
\newcommand{\spec}[1]{S_{#1}(\omega,\tau)}


% Korrespondenzzeichen
\newcommand{\korresp}{\mbox{\setlength{\unitlength}{0.1em}%
                            \begin{picture}(34,10)%
                              \put(10,3){\circle{4}}%
                              \put(12,3){\line(1,0){11}}%
                              \put(24,3){\circle*{4}}%
                            \end{picture}%
                           }%
                     }%
\newcommand{\invkorresp}{\mbox{\setlength{\unitlength}{0.1em}%
                            \begin{picture}(34,10)%
                              \put(10,3){\circle*{4}}%
                              \put(11,3){\line(1,0){11}}%
                              \put(24,3){\circle{4}}%
                            \end{picture}%
                           }%
                     }%
\newcommand{\rotkorresp}{\mbox{\setlength{\unitlength}{0.1em}%
                            \begin{picture}(10,30)%
                              \put(6,8){\circle{4}}%
                              \put(6,10){\line(0,1){11}}%
                              \put(6,22){\circle*{4}}%
                            \end{picture}%
                           }%
                     }%
\newcommand{\rotinvkorresp}{\mbox{\setlength{\unitlength}{0.1em}%
                            \begin{picture}(10,30)%
                              \put(6,8){\circle*{4}}%
                              \put(6,9){\line(0,1){11}}%
                              \put(6,22){\circle{4}}%
                            \end{picture}%
                           }%
                     }%

% Matrix
\newcommand{\mat}[1]{{\mathbf{#1}}}
\newcommand{\dimension}[1]{\operatorname{dim}\left\{ #1 \right\}}


% Integral- und Summengrenzen ober- und unterhalb setzen
\newcommand{\intl}{\int\limits}
\newcommand{\suml}{\sum\limits}
\newcommand{\prodl}{\prod\limits}



% Darstellung eines Bruch mit einem Schr�gstrich
% Makro \nicefrac
%\newcommand{\nicefrac}[2]{\leavevmode\kern.1em\raise.5ex
%                          \hbox{\the \scriptfont0 #1}\kern
%                          -.1em / \kern-.15em\lower.25ex
%                          \hbox{\the \scriptfont0 #2} }

% Rahmen von Formeln
\newcommand{\rahmen}[1]{\boxed{\hspace{0.5cm}\begin{array}{c}#1\end{array}\hspace{0.5cm}}}
