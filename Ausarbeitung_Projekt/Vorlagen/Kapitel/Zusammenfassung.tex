\chapter{Zusammenfassung}
In diesem Abschnitt ist zu beschreiben, welche Arbeiten durchgef�hrt wurden. D.h. die eigene Leistung ist unter Hinweis auf die dabei verwendeten Methoden und Vorgehensweisen hier darzustellen. Dabei ist eine Einordnung der Ergebnisse in das allgemeine Problemumfeld vorzunehmen, das in Einleitung und Stand der Technik zur Sprache kam. Sie sollten hier auf die vorhergehenden Kapitel verweisen, um dem Leser der quer liest, die M�glichkeit zu geben, die Details anzusehen ({\LaTeX } \verb+\ref{...}+).

%---------------------------------------------------------
\section{ToDo}
An dieser Stelle sind die Arbeiten aufzuf�hren, die noch zwingend durchgef�hrt werden sollen. Punkte aus dem Pflichtenheft, die optional waren und Aspekte, die sich aus dem Projekt neu ergeben haben.

%---------------------------------------------------------
\section{Ausblick}
Im Ausblick ist darzustellen, wie das Projekt weitergef�hrt werden kann. Dies kann auch einen konkreten Arbeitsplan enthalten.

%---------------------------------------------------------
\section{Fehlerliste}
In diesem Abschnitt ist anzugeben, ob Fehler in der Arbeit enthalten sind, die nicht mehr beseitigt werde konnten. F�r eine Weiterf�hrung der Arbeit ist dies sehr wichtig.
