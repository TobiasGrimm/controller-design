\chapter{Hauptteil}
Der Hauptteil enth�lt die zentralen Kapitel der Arbeit. In diesen wird die spezielle Fragestellung der Arbeit dargestellt. Es findet die Erl�uterung von Vorgehensweisen und durch sie erzielter Ergebnisse statt. Herleitung der Ergebnisse durch eine logische Gedankenkette. Im Falle einer Informatik orientierten Arbeit zus�tzlich sind Erl�uterung der Implementation durchzuf�hren.

Die Arbeit kann sehr unterschiedliche gegliedert werden. Was sich in den Projekten der Sicherheitstechnologie bew�hrt hat ist die Erl�uterung der einzelnen funktionalen Bl�cke. Dabei wird nicht im groben zwischen Hard- und Software unterschieden, sonder es wird der Block beschrieben und darin der Hardwareaufbau und die entsprechende Art der Implementierung. Also Beispielsweise
%---------------------------------------------------------
\begin{lstlisting}[caption={Beispielstruktur}]
    \section{Eingangssensoren}
    	\subsection{Taster}
    	\subsection{Temperatursensor}
    	\subsection{Implementierung der Eingangssensoren}
    \section{Mikrocontrollereinheit}
    	\subsection{Controller}
    	\subsection{Energieversorgung}
    	\subsection{Elektronik des Netzwerk}
    	\subsection{Implementierung des TCP/IP Stack}
    	\subsection{Testen der Implementierung}
    \section{Leistungsbedarf der Elektronik}
    \section{Kosten der Hardware}
    \section{Inbetriebnahme der Eingageeinheit}   
\end{lstlisting}


Anbei sind einige Abschnitte exemplarisch in Abh�ngigkeit Ihrer Themenstellung angegeben.

%---------------------------------------------------------
\section{Hardware-Design}
In diesem Kapitel ist die Hardware zu beschreiben. Zu Beginn sind die Anforderungen der Hardware anzugeben. Hieraus ist die Struktur zu entwickeln die durch ein Blockschaltbild zu visualisieren ist. Anhand der Struktur kann das Kapitel in Abschnitte unterteilt werden.

Beim Beschreiben der Hardware ist ein detailliertes Blockschaltbild zu entwerfen, hieraus ist der Schaltplan und das Layout zu entwickeln. Die Hardware ist durch Tests und Messungen auf ihre Korrektheit, bezugnehmend auf die Spezifikation, zu untersuchen. Eventuell sind Spezifikationsdaten durch Messungen zu verifizieren. Zum Abschluss sind Fotos der Platine zu erstellen und einzuf�gen. Bilder der Hardware und des Aufbaus sind mit einer Digitalkamera abzulichten und in die Ausarbeitung mit aufzunehmen. Platinen lassen sich am besten fotografieren, wenn Sie auf ein wei�es Blatt Papier gelegt werden.

Die Bauteilauswahl ist anhand der Anforderungen und Verf�gbarkeit genau zu begr�nden. Alternative zu den Bauteilen sollten angegeben werden.

Beispielhaft w�ren folgende Abschnitte:
%---------------------------------------------------------
\begin{lstlisting}[caption={Beispielstruktur Hardware}]
    \subsection{Planung}
    \subsection{Blockschaltbild}
    \subsection{Schaltung}
    \subsection{Layout}
    \subsection{Tests/Messungen}
\end{lstlisting}


%---------------------------------------------------------
\section{Software-Design}
In diesem Kapitel ist die Software zu beschreiben. Zu Beginn sind die Anforderungen der Software anzugeben. Hieraus ist die Struktur zu entwickeln, die durch Zeichnungen (Programmablaufplan, Struktogramm, Zustandsgraph, ...) zu visualisieren ist. Anhand der Struktur kann das Kapitel in Abschnitte unterteilt werden. Eine Top-Down Vorgehensweise hat sich bew�hrt.

Beispielhaft w�ren folgende Abschnitte:
%---------------------------------------------------------
\begin{lstlisting}[caption={Beispielstruktur Software}]
    \subsection{Gesamtstruktur}
    \subsection{Betriebssystem}
    \subsection{Mikrocomputersystems}
    \subsection{Algorithmus zur Ansteuerung der LED's}
\end{lstlisting}

%---------------------------------------------------------
\section{Diskussion der Ergebnisse}
\textquote[{\cite{Hertel:05}}]{In diesem Abschnitt er�rtern Sie, was die Messergebnisse \emph{bedeuten}. Dabei greifen Sie auf vorsorglich bereitgestellte Theorien zur�ck. Beispielsweise k�nnen Sie die Parameter eines vorher beschriebenen Modells anpassen. Manchmal l�sst sich auf Grund der Messdaten zwischen Alternativen entscheiden, die man zuvor formuliert haben sollte.}

%---------------------------------------------------------
\section{Leistungsbewertung}
\textquote[{\cite{dresden:08}}]{Aus diesem Kapitel sollte hervorgehen, welche Methoden angewendet wurden um, die Leistungsf�higkeit zu bewerten und welche Ergebnisse dabei erzielt wurden. Wichtig ist es, dem Leser nicht nur ein paar Zahlen hinzustellen, sondern auch eine Diskussion der Ergebnisse vorzunehmen. Sehr gut ist, wenn man zun�chst diskutiert und plausibel macht, welche Ergebnisse man erwartet, und dann eventuelle Abweichungen diskutiert.}

