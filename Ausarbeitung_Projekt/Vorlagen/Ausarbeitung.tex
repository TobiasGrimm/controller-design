% ----------------------------------------------------------------
% Haupt-Dokument  ************************************************
%-----------------------------------------------------------------
%
% Autor:   Prof. Dr.-Ing. Dirk Benyoucef
% Version: V3.1
% Letzte �nderung : 28.09.2011
%
%-----------------------------------------------------------------
% Optionen des KOMA-Script Paket
%---------------------------------
\documentclass[11pt,                        % Schriftgr��e
               fleqn,                       % Gleichungen linksb�ndig
               halfparskip,                 %
               a4paper,                     % Papierformat A4
               headsepline,                 % einzeichnen einer Linie unter die Kopfzeile
               idxtotoc,                    % Aufnahme des Indexverzeichnis in das Inhaltsverzeichnis
               bibtotoc,                    % Aufnahme des Literaturverzeichnis in das Inhaltsverzeichnis
               liststotoc,
               bibtotocnumbered,            % Nummerierung des Literaturverzeichnis
               pointlessnumbers,            % Bei den Abschnittsnummern wird kein
                                            % Abschlie�ender Punkt gesetzt
               DIV15,                       % Andere Seitenaufteilung
               BCOR1.0cm,                   % Rand zum Binden
               openbib,						% Erzeugt Abs�tze im Literaturverzeichnis
               ngerman,
               ]{scrbook}
%-----------------------------------------------------------------
% Einbinden verschiedener n�tzlicher Pakete
%---------------------------------
\usepackage[ansinew]{inputenc}  % Erm�glicht die Verwendung von Umlaute
                                % und Sonderzeichen: 'ansinew'  f�r Windows
                                % 'latin1' f�r Linux
\usepackage[T1]{fontenc}        % 8-Bit-Fonts erleichtern die Trennung bei Umlauten
\usepackage[ngerman]{babel}     % deutschsprachige Anpassung; deutsche Trennmuster
\usepackage{varioref}
\usepackage{xcolor}             % Farbige Darstellung von W�rtern
\usepackage{framed}
\usepackage{graphicx}           % erm�glicht das Einbinden von Grafikdateien
\usepackage[sf,bf,SF,hang]{subfigure}   % Mehere Bilden in ein Abbildung
\usepackage[sf]{caption}        % Unterschrift in ein Abbildung
\usepackage{here}               % Bilder sollen beim Argument H an Ort und Stelle erscheinen
\usepackage{array}              % F�r komplexe Tabellen
%\usepackage{tabularx}           % Tabellen mit automatischer Berechnung der Spaltenbreite
\usepackage{booktabs}           % Tabellen
\usepackage{longtable}          % Erlaubt Tabellen �ber mehrere Seiten
\usepackage{multirow}           % F�r die Tabellen um mehrere Spalten zusammenzufassen
\usepackage{eurosym}            % Eurosymbol
\usepackage{url}                % Verwenden von URLs
\urlstyle{tt}
\usepackage[geometry]{ifsym}    % F�r Symbole
\usepackage[intlimits,sumlimits]{amsmath} % erweiterte Formeln, Optionen
                                % erm�glichen die Grenzen oberhalb und
                                % unterhalb zu setzen
\usepackage{amssymb}            % blackboard Buchstaben (z.B. Symbol f�r komplexe Zahlen IC)
\usepackage{theorem}            % Em�glicht die Definition von eigenen Umgebungen
\usepackage[amssymb,thickqspace]{SIunits}   % Vereinfacht den Umgang mit Einheiten
\usepackage{icomma}           % setzt das Komma bei Zahlen richtig (besonders bei mathematische Formeln)
\usepackage[pdfpagelabels,
            bookmarksnumbered=true,
            bookmarksopen=true,
            bookmarksopenlevel=1,
			pdftitle={ xxx},
			pdfauthor={ xxx },
			pdfsubject={Anleitung},
            pdfstartview=FitV,
			plainpages=false]{hyperref}
\usepackage{pdfpages}            % Mehrseitige PDFs einbinden
\usepackage[babel]{csquotes}
\usepackage{listings}
\usepackage{makeidx}
\usepackage{fancybox} 	% Boxen definieren und verorten
\usepackage{svn-multi}
\svnid{$Id: Ausarbeitung.tex 169 2011-09-28 14:22:43Z dbenyoucef $}


% Auswahl einer Schrift
\usepackage{cmbright}
%-----------------------------------------------------------------
% Einbinden einiger selbst definierten Makros
%---------------------------------
%
% Einige n�tzliche Makros f�r die Verwendung in Texten
%------------------------------------------------------
%
% Autor:   Dirk Benyoucef
% Version: V1.6
% Letzte �nderung : 55.02.2008
%

% Classe KOMA
%\renewcommand*{\capfont}{\captionfont}
%\renewcommand*{\caplabelfont}{\captionlabelfont}
\renewcommand{\figurename}{Bild}
\renewcommand{\tablename}{Tabelle}


% Umdefinition der Symbole f�r item
\renewcommand{\labelitemiii}{\FilledSmallSquare}
\renewcommand{\labelitemi}{\FilledSmallTriangleRight}
\renewcommand{\labelitemii}{\FilledSmallDiamondshape}

% Laborname
\newcommand{\DCSP}{\emph{Digital Communications \& Signal Processing }}

% Boxen definieren
\theoremheaderfont{\normalfont\bfseries}
\theoremstyle{break}
\theorembodyfont{\slshape}

\newtheorem{satzenv}{Anforderungen}[chapter]

\definecolor{boxgray}{gray}{0.9}

\newenvironment{graubox}{%
  \setlength{\fboxsep}{10pt}%
  \def\FrameCommand{\fcolorbox{black}{boxgray}}%
  \MakeFramed {\advance\hsize-30pt \FrameRestore}}%
 {\endMakeFramed}

\newenvironment{inhalt}%
{%
  \begin{graubox}
  \begin{satzenv}
}{%
  \end{satzenv}
  \end{graubox}
}

\newenvironment{grau_verbatim}%
{%
  \begin{graubox}
  \begin{verbatim}
}{%
  \end{verbatim}
  \end{graubox}
}


% Real- Imagin�rteil
\newcommand{\real}[1]{\operatorname{Re}{\left\{ #1 \right\}}}
\newcommand{\imag}[1]{\operatorname{Im}{\left\{ #1 \right\}}}

% Spezielle Befehle f�r OFDM
\newcommand{\Comment}[1]{}
\newcommand{\complex}[1]{\underline{#1}}
\newcommand{\dft}[1]{\operatorname{DFT}\left\{ #1 \right\}}
\newcommand{\idft}[1]{\operatorname{IDFT}\left\{ #1 \right\}}

\newcommand{\domega}{e^{j 2 \pi f T}}
\newcommand{\aOmega}{e^{j \Omega}}
\newcommand{\vOmega}{\Omega}

%
% STANDARDISIERUNG einiger SYMBOLE
%

% Mengenbuchstaben
%
\newcommand{\mm}[1]{\ensuremath{\mathbb{#1}}}
\newcommand{\mc}{\mm{C}}            % komplexe Zahlen
\newcommand{\mn}{\mm{N}}            % nat�rliche Zahlen
\newcommand{\mg}{\mm{G}}
\newcommand{\mi}{\mm{I}}
\newcommand{\mq}{\mm{Q}}
\newcommand{\mr}{\mm{R}}            % reelle Zahlen
\newcommand{\mw}{\mm{W}}
\newcommand{\mz}{\mm{Z}}            % ganze Zahlen
\newcommand{\rz}{{\mathbb R}}       % reelle Zahlen
\newcommand{\cz}{{\mathbb C}}       % komplexe Zahlen
\newcommand{\zz}{{\mathbb Z}}       % ganze Zahlen
\newcommand{\nz}{{\mathbb N}}       % nat�rliche Zahlen
\newcommand{\Nnull}{{\mathbb N}_0}  % nat�rliche Zahlen ohne die Null

\newcommand{\ma}{\mathcal{A}}            % Menge von A
\newcommand{\mb}{\mathcal{B}}            % Menge von B
\newcommand{\mt}{\mathcal{T}}
\newcommand{\mx}{\mathcal{X}}

% Einige Normen
\newcommand{\norm} [1]{\left\| #1 \right\|}
\newcommand{\lnorm}[1]{\left\| #1 \right\|_{l^2}}
\newcommand{\Lnorm}[1]{\left\| #1 \right\|_{L^2}}
\newcommand{\wnorm}[1]{\left\| #1 \right\|_{\mathrm W}}

% Skalarprodukte
\newcommand{\skalar} [2]{\left\langle #1, #2\right\rangle}
\newcommand{\lskalar}[2]{\left\langle #1, #2\right\rangle_{l^2}}
\newcommand{\Lskalar}[2]{\left\langle #1, #2\right\rangle_{L^2}}

% Erwartungswert
\newcommand{\erw}[1]{\mathcal{E}{\left\{#1\right\}}}
\newcommand{\var}[1]{\operatorname{var}\left\{#1\right\}}

% Vektoralebra
\newcommand{\rot}{\operatorname{rot}}
\newcommand{\divergenz}{\operatorname{div}}
\newcommand{\grad}{\operatorname{grad}}

% Neue Signale oder Operatoren
\newcommand{\dif}{\operatorname{d \!}}
\newcommand{\rect}{\operatorname{rect}}
\newcommand{\erfc}{\operatorname{erfc}}
\newcommand{\erf}{\operatorname{erf}}
\newcommand{\si}{\operatorname{si}}
\newcommand{\Si}{\operatorname{Si}}
\newcommand{\adj}{\operatorname{adj}}
\newcommand{\sign}[1]{\operatorname{sign}\left\{ #1 \right\}}
\newcommand{\spur}[1]{\operatorname{spur}\left\{ #1 \right\}}
\newcommand{\diag}[1]{\operatorname{diag}\left\{ #1 \right\}}
\newcommand{\Per}{\operatorname{Per}}
\newcommand{\Min}{\operatorname{Min}}


% Transformationen
\newcommand{\trafo}[1]{\mathcal{T}{\left\{#1\right\}}}
\newcommand{\four}[1]{\mathcal{F}{\left\{#1\right\}}}
\newcommand{\fourdft}[1]{\mathcal{F}_{DFT}{\left\{#1\right\}}}
\newcommand{\invfour}[1]{\mathcal{F}^{-1}{\left\{#1\right\}}}
\newcommand{\invfourdft}[1]{\mathcal{F}^{-1}_{DFT}{\left\{#1\right\}}}
\newcommand{\stft}[1]{\mathcal{F}{_{STFT}\left\{#1\right\}}}
\newcommand{\ztrafo}[1]{\mathcal{Z}{\left\{#1\right\}}}
\newcommand{\invztrafo}[1]{\mathcal{Z}^{-1}{\left\{#1\right\}}}
\newcommand{\hilbert}[1]{\mathcal{H}\left\{ #1 \right\}}
\newcommand{\hilbertinv}[1]{\mathcal{H}^{-1}\left\{#1 \right\}}
\newcommand{\stftmert}[1]{\mathcal{F}_{#1}^\gamma(\omega,\tau)}
\newcommand{\fourmert}[1]{\mathcal{F}_{#1}(\omega,\tau)}
\newcommand{\wave}[1]{\mathcal{W}{\left\{#1\right\}}}
\newcommand{\wavep}[1]{\mathcal{W}{_{WP}\left\{#1\right\}}}
\newcommand{\spec}[1]{S_{#1}(\omega,\tau)}


% Korrespondenzzeichen
\newcommand{\korresp}{\mbox{\setlength{\unitlength}{0.1em}%
                            \begin{picture}(34,10)%
                              \put(10,3){\circle{4}}%
                              \put(12,3){\line(1,0){11}}%
                              \put(24,3){\circle*{4}}%
                            \end{picture}%
                           }%
                     }%
\newcommand{\invkorresp}{\mbox{\setlength{\unitlength}{0.1em}%
                            \begin{picture}(34,10)%
                              \put(10,3){\circle*{4}}%
                              \put(11,3){\line(1,0){11}}%
                              \put(24,3){\circle{4}}%
                            \end{picture}%
                           }%
                     }%
\newcommand{\rotkorresp}{\mbox{\setlength{\unitlength}{0.1em}%
                            \begin{picture}(10,30)%
                              \put(6,8){\circle{4}}%
                              \put(6,10){\line(0,1){11}}%
                              \put(6,22){\circle*{4}}%
                            \end{picture}%
                           }%
                     }%
\newcommand{\rotinvkorresp}{\mbox{\setlength{\unitlength}{0.1em}%
                            \begin{picture}(10,30)%
                              \put(6,8){\circle*{4}}%
                              \put(6,9){\line(0,1){11}}%
                              \put(6,22){\circle{4}}%
                            \end{picture}%
                           }%
                     }%

% Matrix
\newcommand{\mat}[1]{{\mathbf{#1}}}
\newcommand{\dimension}[1]{\operatorname{dim}\left\{ #1 \right\}}


% Integral- und Summengrenzen ober- und unterhalb setzen
\newcommand{\intl}{\int\limits}
\newcommand{\suml}{\sum\limits}
\newcommand{\prodl}{\prod\limits}



% Darstellung eines Bruch mit einem Schr�gstrich
% Makro \nicefrac
%\newcommand{\nicefrac}[2]{\leavevmode\kern.1em\raise.5ex
%                          \hbox{\the \scriptfont0 #1}\kern
%                          -.1em / \kern-.15em\lower.25ex
%                          \hbox{\the \scriptfont0 #2} }

% Rahmen von Formeln
\newcommand{\rahmen}[1]{\boxed{\hspace{0.5cm}\begin{array}{c}#1\end{array}\hspace{0.5cm}}}
               % Sammlung von Abk�rzungen f�r Mathematiksymbolen
                                % (ben�tigt die Pakete amssymb und theorem)
\lstset{basicstyle=\sffamily,language=[Latex]Tex,frame=tb,numbers=left, numberstyle=\tiny, emphstyle=\color{red}, columns=fullflexible, backgroundcolor=\color{boxgray}}

\renewcommand{\labelitemiii}{\FilledSmallSquare}
\renewcommand{\labelitemi}{\FilledSmallTriangleRight}
\renewcommand{\labelitemii}{\FilledSmallDiamondshape}


% Auszufuehrende Befehle  ------------------------------------------------
\makeindex

% TITEL
\newcommand{\SEAT}{
\emph{Smart Metering: Disaggregation von Endverbrauchern }}
\newcommand{\SEA}{\emph{SmartMeter }}


%#################################################################
% Titelseite
% ----------------------------------------------------------------
% Einen Titel ausw�hlen
%\titlehead{\vspace{-2cm}\includegraphics[height=15mm]{Bilder/Logo/DCSP_Lab_v1_1_grau.pdf}
           \hfill \includegraphics[height=20mm]{Bilder/Logo/Logo_HFU_rz_4c.pdf}\\
            \textsf{Hochschule Furtwangen}\\
            \textsf{Digital Communication \& Signal Processing Lab}\\
            \textsf{Prof. Dr.-Ing. Dirk Benyoucef}}

\subject{\textsf{Master Thesis}}
\title{Titel}
\author{Names des Studenten}
\date{\today}

\publishers{
    \begin{minipage}{\textwidth}
        \vskip 6cm
         {\normalsize }
         \begin{tabular}{ll}
            Betreuender Hochschullehrer: & Prof. Dr.-Ing. Dirk Benyoucef\tabularnewline
            Betreuender Mitarbeiter:& [Your tutor]\tabularnewline
            Tag der Anmeldung:& 01.03.2008 \tabularnewline
            Tag der Abgabe:   & 30.09.2008\tabularnewline
        \end{tabular}
        {\normalsize }
        \end{minipage}
    }
\extratitle{} 
%\include{Titel_Bachelor}
\titlehead{\vspace{-2cm}
           \hfill \includegraphics[height=20mm]{Bilder/Logo/Logo_HFU_rz_4c.pdf}\\
            \textsf{}\\
            \textsf{}\\
            \textsf{}}

\subject{\textsf{WPV C\# \& Design Patterns}}      % Art der Arbeit
\title{Live-Tuner}
\author{Alexander Baitinger, Tobias Grimm}
\date{\today}

\publishers{
    \begin{minipage}{\textwidth}
        \vskip 6cm
         {\normalsize }
         \begin{tabular}{ll}
            Betreuer der Hochschule: & Prof. Dr. Gerd Unruh\tabularnewline
        \end{tabular}
        {\normalsize }
        \end{minipage}
    }
\extratitle{} 

%#################################################################
% Begin des Document
% ----------------------------------------------------------------
\begin{document}
 
    \fancyput*(-1.8cm,-26cm){			 % Titelzeile auf allen Seiten
    	\rotatebox{90}{
    		\color{gray}
    		\footnotesize
    		\sf{DCSP-Lab, Prof. Dr.-Ing. Dirk Benyoucef, \today, Rev: \svnrev}
    	}
    }

    \pagenumbering{roman}
    \maketitle                        % Erzeugen der Titelseite und der Widmung
    \tableofcontents                  % Erzeugen des Inhaltsverzeichnisses

    \frontmatter
    \chapter*{Kurzfassung}
Die Kurzfassung sollte die Zielsetzung der Arbeit sowie ihre wesentlichen Ergebnisse und Erkenntnisse enthalten und eine L�nge von maximal 20 Zeilen
haben. Diese ist in deutsch und englisch zu verfassen. Die Kurzfassung ist f�r bibliothekarische Zwecke notwendig.

F�nf Punkte der Kurzfassung
\begin{enumerate}
    \item Eingrenzung des Arbeits- bzw. Forschungsbereichs (In  welchem  Themengebiet ist die Arbeit angesiedelt?)
    \item Beschreibung des Problems (Was ist das Problem und warum ist es wichtig dies zu l�sen?)
    \item M�ngel an existierenden Arbeiten bzgl. des Problems (Warum ist es ein Problem, obwohl sich schon andere mit dem gleichen Thema besch�ftigt haben?)
    \item Eigener L�sungsansatz (Welcher Ansatz wurde in dieser Arbeit verwendet, um das Problem zu l�sen? Was ist der Beitrag dieses Artikels?)
    \item Art der Validierung + Ergebnisse (Wie wurde nachgewiesen, dass die Arbeit die versprochene Verbesserung wirklich vollbringt (Fallstudie, Experiment, o.�.); Was waren die Ergebnisse der Validierung (idealerweise Prozentsatz der Verbesserung)?)
\end{enumerate}


\begin{graubox}
\textbf{Kurzfassung}

Text Text Text Text Text Text Text Text Text Text Text Text Text Text
Text Text Text Text Text Text Text Text Text Text Text Text Text Text
Text Text Text Text Text Text Text Text Text Text Text Text Text Text
Text Text Text Text Text Text Text Text Text Text Text Text Text Text
Text Text Text Text Text Text Text Text Text Text Text Text Text Text
Text Text Text Text Text Text Text Text Text Text Text Text Text Text
\end{graubox}


\begin{graubox}
\textbf{Abstract}

Text Text Text Text Text Text Text Text Text Text Text Text Text Text
Text Text Text Text Text Text Text Text Text Text Text Text Text Text
Text Text Text Text Text Text Text Text Text Text Text Text Text Text
Text Text Text Text Text Text Text Text Text Text Text Text Text Text
Text Text Text Text Text Text Text Text Text Text Text Text Text Text
Text Text Text Text Text Text Text Text Text Text Text Text Text Text
\end{graubox}


    \chapter*{Personen des Projekts}

%---------------------------------------
\begin{table}[h]
  \sffamily
  \centering
  \small
  %\caption{Bearbeiter des Projektes}
  \label{tab:Personen}
  \begin{tabular}{m{4cm} m{3cm} m{8cm} }
    \toprule
    Name & Foto & Curriculum Vita\\
   \midrule
      Klaus Muster\newline 
      Projektleiter
                        & \includegraphics[width=3cm]{Bilder/Logo/DCSP_Lab_v1_1_grau.pdf} & Abitur\\
   \cmidrule(lr){1-3}
      Otto Musterfrau\newline
      Schriftf�hrer
                        & \includegraphics[width=3cm]{Bilder/Logo/DCSP_Lab_v1_1_grau.pdf} & Berufsausbildung als Industieelektroniker,\newline 
                                                                                            Fachabitur in Schwennigen\\
   \cmidrule(lr){1-3}
      Klaus Muster      & \includegraphics[width=3cm]{Bilder/Logo/DCSP_Lab_v1_1_grau.pdf} & Berufsausbildung als Maurer, Abitur\\
   \bottomrule
  \end{tabular}
\end{table}
    \cleardoublepage                  % zum Beenden der roman Seitennummerierung! \clearpage
    \pagenumbering{arabic}            % Seitennummerierung Hauptteil

    \cleardoublepage

    % Hauptteil
    \mainmatter

    %#################################################################
    % Kapitel
    % ----------------------------------------------------------------
    \chapter{Einleitung}

Das 6. Semester der Hochschule Furtwangen beinhaltet f�r den Studiengang Elektronik und Technische Informatik (ETI) die M�glichkeit durch Wahlpflicht Vorlesungen (WPVs) das gesamt Bild des angehenden Engineers abzurunden.

Gerade f�r einen Elektrotechniker bietet es sich daher an, nicht nur die Elektrotechnische-Welt zu kennen, sondern auch die unz�hligen M�glichkeiten der Informatik kennen zu lernen, und diese in Kombination zu nutzen.

Aus diesem Gedanken heraus ist die Idee entstanden ein Programm mit C\# zu entwickeln, welches den Arbeitsalltag eines Elektrotechnikers enorm erleichtern kann, indem regelungstechnische Auslegungen schnell und komfortabel durchgef�hrt werden k�nnen.

Um beide Welten sauber in Einklang bringen zu k�nnen wird auf der Elektrotechnischen Seite auf die Regelungstechnik mit der Laplace-Transformation und die Z-Transformation zur�ck gegriffen und auf der Informatik Seite auf die Vorz�ge der Objektorientierten Programmiersprache C\# und die Verwendung von "`Design Patterns"'.

%---------------------------------------------------------
\section{Wichtiges �ber die Regelungstechnik}
\begin{description}

\item \textbf{Warum Regelungstechnik?}

In sehr vielen Anwendungen kommt es vor, dass man einen soll-Wert vorgibt, und diesen mit einem ist-Wert vergleicht und dann entscheidet was gemacht werden soll. Dieser Vorgang wird in der Regelungstechnik behandelt.

So kann man sich z.B. vorstellen, dass ein Motor auf eine gewisse soll-Drehzahl beschleunigt werden soll. Da die Drehzahl sich wegen der Tr�gheit der Masse des Motors nicht direkt auf die gew�nschte Drehzahl begibt, wird ein Regler eingesetzt. Dieser schaut sich den Soll-Ist Vergleich an und entscheidet dann, ob entweder \textit{mehr Drehmoment} oder \textit{weniger Drehmoment} aufgebracht werden muss, um den Motor auf die gew�nschte Drehzahl zu bringen.

\begin{figure}[H]
	\centering
  \includegraphics[width=1\textwidth]{Bilder3/Motor_Regelkreis.pdf}
	\caption{Beispiel Regelkreis f�r einen Motor}
	\label {Beispiel Regelkreis f�r einen Motor}
\end{figure}

In der Regelungstechnik werden nun verfahren untersucht, um diesen Regler ideal einzustellen, damit dass zeitlich dynamische Verhalten eines Vorgangs ideal in den Griff gebracht werden kann.


\item \textbf{Was sind �bertragungsfunktion?}

Um dynamische Vorg�nge in der Physikalischen Welt Mathematisch beschreiben zu k�nnen werden in der Regel so genannte Differentialgleichungen eingesetzt.
Hierdurch kann z.B. das dynamische Verhalten eines Motors, einer elektronischen Schaltung, einer Temperatur,eines mechanischen Vorgangs, ... beschrieben werden.

Das eigentlich Problem besteht nun oft darin, dass das l�sen dieser Differentialgleichnung sehr schwer ist. 
An dieser Stelle bietet die Laplace-Transformation einen eleganten Weg, um aus einer Differentialgleichung eine �bertragungsfunktion im Laplace-Bereich zu gewinnen.

Diese �bertragungsfunktion (Transferfunktion) bietet f�r unsere Zwecke entscheidende Vorteile:

\begin{itemize}
	\item \textit{Handhabung}
	
	Es ist wesentlich einfacher mit einer �bertragungsfunktion zu rechnen.
	\item \textit{Universell}
	
	Die Regelungstechnik ist f�r alle dynamischen Vorg�nge equivalent!
	\item \textit{Simulierbar}
	
	Durch Verwendung der Z-Transformation kann eine �bertragungsfunktion "`relativ"' einfach auf einem Computer Simuliert werden.
	\item \textit{Betrachtung als Block}
	
	Die einzelnen �bertragungsfunktionen f�r Regler, Motor, ... k�nnen als Bl�cke wie in Bild\vref{Beispiel Regelkreis f�r einen Motor} betrachtet werden.
\end{itemize}

\newpage
\item \textbf{Warum ein extra Software Tool?}

Die gesamte Regelungstechnik besteht zum gr��ten Teil aus Mathematik. Hier kann es sehr schnell passieren, dass man den �berblick verliert, da es einem schwer f�llt gewisse abstrakte Gebilde sich vorzustellen. 

Das hier entwickelte Software Tool m�chte genau an dieser Stelle ans�tzen, dem Benutzer ein Gef�hl daf�r zu geben, wie die einzelnen abstrakten Gebilde zusammen h�ngen, und welchen Einfluss die einzelnen Parameter auf das dynamische Verhalten des Systems haben.

Dar�ber hinaus sind bereits verfahren hinterlegt, um einen Regler gut auszulegen, und diesen visuell in Echtzeit an die Bed�rfnisse des Engineers nach tunen zu k�nnen.
Aus diesem grund hat das neu entstandene Tool den Namen "`Live-Tuner"' erhalten.

\end{description}
    \section{Systemkonzept (Adam Visy)}

In diesem Kapitel wird das Konzept erkl�rt, mit dem  das Problem aus der Aufgabenstellung gel�st wurde.
Die Abbildung \vref{fig:syskonzept} dient zur Veranschaulichung des L�sungswegs der gegebenen Aufgabenstellung. Der Kern des eingebetteten Systems besteht aus demPIC32MX795F512L der Firma Microchip, sowie aus einer auf das DCSP11 Board basierten Platine.

Bei der Bet�tigung des Tasters wechselt das System vom Ruhezustand in den normalen Betrieb. Hier k�nnen nun verschiedene Ausgaben �ber die Stroboskopuhr dargestellt werden. Die vom DCF77-Empfangsmodul ermittelte Uhrzeit wird bei Bedarf mit der internen Uhrzeit durch den RTCC synchronisiert und gegebenenfalls �ber die RGB LEDs ausgegeben. Die RGB LEDs werden �ber drei TLCs, die �ber einen I2C-Bus mit dem PIC kommunizieren, angesprochen. Eine weitere Ausgabem�glichkeit, f�r Wartungszwecke, stellt das Display dar. Der Summer erm�glicht eine akustische Ausgabe f�r zum Beispiel die Weckfunktion. Das Pendel wird von einem Gleichstrommotor der Firma Dunkermotoren bewegt. Die Drehzahl wird  analog vom Mikrocontroller vorgegeben. Die Position des Zeigers wird mithilfe eines, auf dem Pendel befindenden, Hallsensors festgestellt. �ber einen Photowiderstand wird die Umgebungslichtst�rke gemessen. So kann die Lichtst�rke der RGB-LEDs angepasst werden. Der Webserver soll �ber die Ethernetschnittstelle von au�en angesprochen werden. Der Webserver wird auf einer mit ebenfalls SPI-Bus angebundenen SD-Karte hinterlegt.

\begin{figure}[htbp]
	\centering
		\includegraphics[width=0.8\textwidth]{Bilder/syskonzept3.pdf}
	\caption{Blockschaltbild des Sysemkonzepts}
	\label{fig:syskonzept}
\end{figure}



    \chapter{Hauptteil}
Der Hauptteil enth�lt die zentralen Kapitel der Arbeit. In diesen wird die spezielle Fragestellung der Arbeit dargestellt. Es findet die Erl�uterung von Vorgehensweisen und durch sie erzielter Ergebnisse statt. Herleitung der Ergebnisse durch eine logische Gedankenkette. Im Falle einer Informatik orientierten Arbeit zus�tzlich sind Erl�uterung der Implementation durchzuf�hren.

Die Arbeit kann sehr unterschiedliche gegliedert werden. Was sich in den Projekten der Sicherheitstechnologie bew�hrt hat ist die Erl�uterung der einzelnen funktionalen Bl�cke. Dabei wird nicht im groben zwischen Hard- und Software unterschieden, sonder es wird der Block beschrieben und darin der Hardwareaufbau und die entsprechende Art der Implementierung. Also Beispielsweise
%---------------------------------------------------------
\begin{lstlisting}[caption={Beispielstruktur}]
    \section{Eingangssensoren}
    	\subsection{Taster}
    	\subsection{Temperatursensor}
    	\subsection{Implementierung der Eingangssensoren}
    \section{Mikrocontrollereinheit}
    	\subsection{Controller}
    	\subsection{Energieversorgung}
    	\subsection{Elektronik des Netzwerk}
    	\subsection{Implementierung des TCP/IP Stack}
    	\subsection{Testen der Implementierung}
    \section{Leistungsbedarf der Elektronik}
    \section{Kosten der Hardware}
    \section{Inbetriebnahme der Eingageeinheit}   
\end{lstlisting}


Anbei sind einige Abschnitte exemplarisch in Abh�ngigkeit Ihrer Themenstellung angegeben.

%---------------------------------------------------------
\section{Hardware-Design}
In diesem Kapitel ist die Hardware zu beschreiben. Zu Beginn sind die Anforderungen der Hardware anzugeben. Hieraus ist die Struktur zu entwickeln die durch ein Blockschaltbild zu visualisieren ist. Anhand der Struktur kann das Kapitel in Abschnitte unterteilt werden.

Beim Beschreiben der Hardware ist ein detailliertes Blockschaltbild zu entwerfen, hieraus ist der Schaltplan und das Layout zu entwickeln. Die Hardware ist durch Tests und Messungen auf ihre Korrektheit, bezugnehmend auf die Spezifikation, zu untersuchen. Eventuell sind Spezifikationsdaten durch Messungen zu verifizieren. Zum Abschluss sind Fotos der Platine zu erstellen und einzuf�gen. Bilder der Hardware und des Aufbaus sind mit einer Digitalkamera abzulichten und in die Ausarbeitung mit aufzunehmen. Platinen lassen sich am besten fotografieren, wenn Sie auf ein wei�es Blatt Papier gelegt werden.

Die Bauteilauswahl ist anhand der Anforderungen und Verf�gbarkeit genau zu begr�nden. Alternative zu den Bauteilen sollten angegeben werden.

Beispielhaft w�ren folgende Abschnitte:
%---------------------------------------------------------
\begin{lstlisting}[caption={Beispielstruktur Hardware}]
    \subsection{Planung}
    \subsection{Blockschaltbild}
    \subsection{Schaltung}
    \subsection{Layout}
    \subsection{Tests/Messungen}
\end{lstlisting}


%---------------------------------------------------------
\section{Software-Design}
In diesem Kapitel ist die Software zu beschreiben. Zu Beginn sind die Anforderungen der Software anzugeben. Hieraus ist die Struktur zu entwickeln, die durch Zeichnungen (Programmablaufplan, Struktogramm, Zustandsgraph, ...) zu visualisieren ist. Anhand der Struktur kann das Kapitel in Abschnitte unterteilt werden. Eine Top-Down Vorgehensweise hat sich bew�hrt.

Beispielhaft w�ren folgende Abschnitte:
%---------------------------------------------------------
\begin{lstlisting}[caption={Beispielstruktur Software}]
    \subsection{Gesamtstruktur}
    \subsection{Betriebssystem}
    \subsection{Mikrocomputersystems}
    \subsection{Algorithmus zur Ansteuerung der LED's}
\end{lstlisting}

%---------------------------------------------------------
\section{Diskussion der Ergebnisse}
\textquote[{\cite{Hertel:05}}]{In diesem Abschnitt er�rtern Sie, was die Messergebnisse \emph{bedeuten}. Dabei greifen Sie auf vorsorglich bereitgestellte Theorien zur�ck. Beispielsweise k�nnen Sie die Parameter eines vorher beschriebenen Modells anpassen. Manchmal l�sst sich auf Grund der Messdaten zwischen Alternativen entscheiden, die man zuvor formuliert haben sollte.}

%---------------------------------------------------------
\section{Leistungsbewertung}
\textquote[{\cite{dresden:08}}]{Aus diesem Kapitel sollte hervorgehen, welche Methoden angewendet wurden um, die Leistungsf�higkeit zu bewerten und welche Ergebnisse dabei erzielt wurden. Wichtig ist es, dem Leser nicht nur ein paar Zahlen hinzustellen, sondern auch eine Diskussion der Ergebnisse vorzunehmen. Sehr gut ist, wenn man zun�chst diskutiert und plausibel macht, welche Ergebnisse man erwartet, und dann eventuelle Abweichungen diskutiert.}


    \chapter{Zusammenfassung}
In diesem Abschnitt ist zu beschreiben, welche Arbeiten durchgef�hrt wurden. D.h. die eigene Leistung ist unter Hinweis auf die dabei verwendeten Methoden und Vorgehensweisen hier darzustellen. Dabei ist eine Einordnung der Ergebnisse in das allgemeine Problemumfeld vorzunehmen, das in Einleitung und Stand der Technik zur Sprache kam. Sie sollten hier auf die vorhergehenden Kapitel verweisen, um dem Leser der quer liest, die M�glichkeit zu geben, die Details anzusehen ({\LaTeX } \verb+\ref{...}+).

%---------------------------------------------------------
\section{ToDo}
An dieser Stelle sind die Arbeiten aufzuf�hren, die noch zwingend durchgef�hrt werden sollen. Punkte aus dem Pflichtenheft, die optional waren und Aspekte, die sich aus dem Projekt neu ergeben haben.

%---------------------------------------------------------
\section{Ausblick}
Im Ausblick ist darzustellen, wie das Projekt weitergef�hrt werden kann. Dies kann auch einen konkreten Arbeitsplan enthalten.

%---------------------------------------------------------
\section{Fehlerliste}
In diesem Abschnitt ist anzugeben, ob Fehler in der Arbeit enthalten sind, die nicht mehr beseitigt werde konnten. F�r eine Weiterf�hrung der Arbeit ist dies sehr wichtig.


    %#################################################################
    % Anhang
    % ----------------------------------------------------------------
    \appendix
    \chapter{Anhang}
Beim Anfertigen der Arbeit erstellte Unterlagen, die nicht zur eigentlichen Darstellung der Arbeit geh�ren, aber dennoch im weiteren Sinne zur Ausarbeitung z�hlen, z.B.
\begin{itemize}
  \item langwierige Herleitungen von Ergebnissen, die in der Arbeit eine untergeordnete Rolle spielen,
  \item Programmcode,
  \item zus�tzliches Bildmaterial (wie z.B. hardcopies von Benutzungsschnittstellen),
  \item Protokolle und �hnliches.
\end{itemize}



    \chapter{Erkl�rung der selbst�ndigen Anfertigung}

\begin{graubox}
    Hiermit erkl�re ich, dass ich die vorliegende Arbeit selbst�ndig verfasst und nur die angegebenen Quellen und Hilfsmittel benutzt habe.

    Gleichzeitig erteile ich dem Labor f�r \DCSP (Leitung Prof. Dr.-Ing. Dirk Benyoucef) der Hochschule Furtwangen ein nicht ausschlie�liches, zeitlich unbegrenztes und unwiderrufliches Nutzungsrecht an den Ergebnissen meiner Diplomarbeit.

    \vspace{2cm}
    Ort, Datum, Unterschrift
\end{graubox}


    \chapter{Nomenklatur}
    \section{Abk�rzungen}
        \renewcommand\arraystretch{1.2}  % Multiplikationsfaktor f�r die
                                 % Zeilenumbr�che in der Tabelle
\begin{longtable}[t]{p{2.5cm} p{13cm}<{\raggedright}}
  A/D, D/A          & Analog/Digital bzw. Digital/Analog\\
  AKF               & Autokorrelationsfunktion\\
  AWGN              & additives, wei�es, Gau�sches Rauschen (additive white Gaussian noise)\\[2mm]

  BER               & Bitfehlerrate (bit error rate)\\[2mm]

  CDMA              & Code Division Multiple Access\\[2mm]

  DFT               & Discrete Fourier Transformation \\
  DMT               & Discrete Multi Ton\\[2mm]

  FDMA              & Frequency Division Multiple Access\\
  FDD               & Frequency Division Duplex\\
  FFT               & Fast Fourier Transformation \\[2mm]

  WMF               & Whitening Matched Filter\\[2mm]
\end{longtable}

    \section{Symbole und Formelzeichen}
        %\renewcommand\arraystretch{1.5}  % Multiplikationsfaktor f�r die
                                 % Zeilenumbr�che in der Tabelle
\begin{longtable}[t]{p{4cm} p{11.5cm}<{\raggedright}}
\multicolumn{2}{l}{\textbf{Mengen und Intervalle}}\\
  $\mn, \Nnull$               & Menge der nat�rlichen Zahlen, $\Nnull = \nz\cap\{0\}$ \\
  $\mz$                       & Menge der ganzen Zahlen \\
  $\mq$                       & Menge der rationalen Zahlen \\
  $\mr$                       & Menge der reellen Zahlen \\
  $\mc$                       & Menge der komplexen Zahlen \\
  $\ma$                       & Signalvorrat, Menge der Amplitudenkoeffizienten \\
  $L^2(\rz)$                  & Raum der quadratisch integrablen Funktionen �ber
                                \mbox{ $L^2(\rz) = \left\{ f:\rz \rightarrow \rz\ \ |\quad \int\limits_{\rz} |f(t)|^2 < \infty \right\}$}\\
  $l^2(\zz)$                  & Raum der quadratisch summierbaren Folgen �ber
                                 \mbox{ $l^2(\zz) = \left\{ c:\zz \rightarrow \rz\ \ |\quad \sum\limits_{k\, \in\, \zz} |c(k)|^2 < \infty \right\}$}\\

\multicolumn{2}{l}{\textbf{Konstante Gr��en}}\\
  $e$                         & Eulersche Zahl, $e \approx 2,71828183$\\
  $j$                         & imagin�re Einheit, $j^2 = -1$\\
  $\pi$                       & Kreiszahl, $\pi \approx 3,141526535$\\
  $\infty$                    & Unendlich\\

\multicolumn{2}{l}{\textbf{Transformationen und Operatoren}}\\
  $\four{\cdot}$              & Fourier--Transformation \\
  $\invfour{\cdot}$           & inverse Fourier--Transformation \\
  $x(t) \korresp X(f)$        & Korrespondenz der Fourier--Transformation\\
  $x(t) \korresp \four{x(t)}$ & Korrespondenz der Fourier--Transformation\\
  $\four{x(k)}$               & Fourier--Transformierte der Folge $x(k)$\\
  $X(\domega)$                & Fourier--Transformierte der Folge $x(k)$\\
  $\ztrafo{\cdot}$            & $z$--Transformation \\
  $\invztrafo{\cdot}$         & inverse $z$--Transformation \\
  $\ztrafo{x(k)}$             & $z$--Transformierte der Folge $x(k)$\\
  $X(z)$                      & $z$--Transformierte der Folge $x(k)$\\
  $\erw{\cdot}$               & Erwartungswertoperator\\
  $\real{\cdot}$              & Realteil einer komplexen Gr��e\\
  $\imag{\cdot}$              & Imagin�rteil einer komplexen Gr��e\\
  $\min\{\cdot\}$             & Minimum einer Gr��e\\
  $\max\{\cdot\}$             & Maximum einer Gr��e\\
  $\spur{\cdot}$              & Spur, Summe der Diagonalelemente einer Matrix\\
  $\diag{\cdot}$              & Diagonalmatrix\\
  $\cdot^*$                   & konjugiert komplex \\
  $\cdot^T$                   & transponiert\\
  $\cdot^H$                   & hermitisch, $\mat{A}^H = (\mat{A}^*)^T$\\
  $|\cdot|$                   & Betrag einer Zahl \\
  $\skalar{\cdot}{\cdot}$     & Skalarprodukt \\
  $\Lskalar{\cdot}{\cdot}$    & Skalarprodukt des Raumes  $L^2(\mr)$,\\
                              & \mbox{$\Lskalar{x}{y} :=\intl_{\mr} x(t)\ y^*(t) dt$} \quad
                                mit $x, y \in L^2(\mr)$\\[3mm]
  $\lskalar{\cdot}{\cdot}$    & Skalarprodukt des Raumes  $l^2(\zz)$,\\
                              & \mbox{$\lskalar{x}{y} :=\sum\limits_{k\, \in\, \zz} (x(k)\ y^*(k))$} \quad
                                mit $x, y \in l^2(\zz)$\\[3mm]
  $\norm{\cdot}$              & Norm \\
  $\lnorm{\cdot}$             & Norm des Raumes $l^2(\mz)$,\\
                              & \mbox{$\lnorm{x} := \sqrt{\lskalar{x}{x}}$} \quad
                                mit $x \in l^2(\mz)$\\[3mm]
  $\Lnorm{\cdot}$             & Norm des Raumes $L^2(\mr)$,\\
                              & \mbox{$\Lnorm{x} := \sqrt{\Lskalar{x}{x}}$} \quad
                                mit $x \in L^2(\mr)$\\[3mm]
\multicolumn{2}{l}{\textbf{Skalare}}\\
  $t$                         & kontinuierliche Zeit, $t \in \mr$, kontinuierliche Zeitparameter \\
  $E_b$                       & Energie pro Bit\\
  $E_s$                       & Energie pro Symbol\\
  $f_A$                       & Abtastfrequenz\\
  $T_A$                       & Abtastintervall\\
  $N_A$                       & Anzahl der Abtastwerte\\
  $k$                         & ganzzahlige Variable, $k \in \mz$, diskreter Zeitparameter \\
  $N_s$                       & Anzahl der Untertr�ger\\
  $N_g$                       & Anzahl der Abtastwerte des Guard--Intervalls\\
  $N  $                       & Anzahl aller Abtastwerte ($N = N_s + N_g$)\\
  $N_h$                       & Anzahl der Abtastwerte des Kanals\\
  $P_{e}$                     & Fehlerwahrscheinlichkeit\\
  $N_0$                       & konstante Rauschleistungsdichte\\
    \\
\multicolumn{2}{l}{\textbf{Funktionen}}\\
  $x(t)$                      & Zeitsignal, $x(t) \in L^2(\mr)$\\
  $s_{tx}(t)$                 & Sendesignal\\
  $s_{rx}(t)$                 & Empfangssignal\\
  $H_{rx}(f)$                 & �bertragungsfunktion des Empfangsfilters\\
  $H(z)$                      & $z$-Transformierte der Folge $h(k)$\\
  $H(e^{j 2 \pi f T})$        & periodische �bertragungsfunktion der zeitdiskreten
                                Impulsantwort $h(k)$ des Gesamtsystems  \\
  $S_{nn}(f)$                 & Leistungsdichtespektrum der Rauschgr��e $n(t)$\\
  $S_{nn}(\domega)$           & periodisches Leistungsdichtespektrum der zeitdiskreten Rauschgr��e $n(k)$\\
  $\rect(t)$                  & Rechteckimpuls der H�he 1 und Dauer $T$\\
  $\erf(x)$                   & Error-Funktion\\
  $\erfc(x)$                  & Komplement�re Error-Funktion\\
  \\
\multicolumn{2}{l}{\textbf{Folgen}}\\
  $x[k]$                      & diskrete Zeitfolge, $x[k] \in l^2(\mz)$\\
  $s_{tx}[k]$                 & Kanalsymbole \\
  $n[k]$                      & zeitdiskretes Rauschsignal\\
    \\
\multicolumn{2}{l}{\textbf{Vektoren}}\\
  $\vec{e}$                   & Fehlervektor\\
  $\vec{s}$                   & Sendesymbolvektor\\
  $\vec{\hat{s}}$             & Sch�tzvektor der Sendesymbole\\
  $\vec{s}_{tx}$              & Sendevektor\\
  $\vec{s}_{rx}$              & Empfangsvektor\\
  $\vec{r}$                   & Empfangssymbolvektor\\
  $\vec{h}$                   & Vektor mit den Abtastwerten der Kanalimpulsantwort\\
  $\vec{n}$                   & Rauschvektor f�r farbiges Rauschen\\
  $\vec{w}$                   & Rauschvektor f�r gau�sches wei�es Rauschen\\
  \\
\multicolumn{2}{l}{\textbf{Matrizen}}\\
  $\mat{D}_{\{\cdot\}}$       & Diagonalmarix\\
  $\mat{R}_{\{\cdot \cdot\}}$ & Kreuz-- oder Autokorrelationsmatrix\\
  $\mat{I}$                   & Einheitsmatrix\\
  $\mat{W}$                   & Fouriermatrix\\
\makebox[4cm]{   } &  \\
\end{longtable}


    \bibliographystyle{Alphadin}    % Alphabetisch sortiertes Literaturverzeichnis mit
                                    % Einordnungsmarken aus Verfasser und
                                    % Erscheinungsjahrk�rzeln
    \bibliography{Literaturstellen} % erzeuge Literaturverzeichnis
                                    % benutze Literaturdatenbank
                                    % Literaturstellen d.h. die Dateien diplom.bib

    % Abbildungs- und Tabellenverzeichnis
    \listoffigures
    \listoftables
    \renewcommand{\lstlistlistingname}{Listingverzeichnis}
    \lstlistoflistings

    \printindex

\end{document} 