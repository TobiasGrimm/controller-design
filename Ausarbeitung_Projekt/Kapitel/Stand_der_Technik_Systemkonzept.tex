\section{Systemkonzept (Adam Visy)}

In diesem Kapitel wird das Konzept erkl�rt, mit dem  das Problem aus der Aufgabenstellung gel�st wurde.
Die Abbildung \vref{fig:syskonzept} dient zur Veranschaulichung des L�sungswegs der gegebenen Aufgabenstellung. Der Kern des eingebetteten Systems besteht aus demPIC32MX795F512L der Firma Microchip, sowie aus einer auf das DCSP11 Board basierten Platine.

Bei der Bet�tigung des Tasters wechselt das System vom Ruhezustand in den normalen Betrieb. Hier k�nnen nun verschiedene Ausgaben �ber die Stroboskopuhr dargestellt werden. Die vom DCF77-Empfangsmodul ermittelte Uhrzeit wird bei Bedarf mit der internen Uhrzeit durch den RTCC synchronisiert und gegebenenfalls �ber die RGB LEDs ausgegeben. Die RGB LEDs werden �ber drei TLCs, die �ber einen I2C-Bus mit dem PIC kommunizieren, angesprochen. Eine weitere Ausgabem�glichkeit, f�r Wartungszwecke, stellt das Display dar. Der Summer erm�glicht eine akustische Ausgabe f�r zum Beispiel die Weckfunktion. Das Pendel wird von einem Gleichstrommotor der Firma Dunkermotoren bewegt. Die Drehzahl wird  analog vom Mikrocontroller vorgegeben. Die Position des Zeigers wird mithilfe eines, auf dem Pendel befindenden, Hallsensors festgestellt. �ber einen Photowiderstand wird die Umgebungslichtst�rke gemessen. So kann die Lichtst�rke der RGB-LEDs angepasst werden. Der Webserver soll �ber die Ethernetschnittstelle von au�en angesprochen werden. Der Webserver wird auf einer mit ebenfalls SPI-Bus angebundenen SD-Karte hinterlegt.

\begin{figure}[htbp]
	\centering
		\includegraphics[width=0.8\textwidth]{Bilder/syskonzept3.pdf}
	\caption{Blockschaltbild des Sysemkonzepts}
	\label{fig:syskonzept}
\end{figure}


