\chapter{Zusammenfassung}

Mit Hilfe dieses Tools k�nnen nun schnell die passenden Werte f�r einen Regler zu einer bestimmten Strecke ermittelt werden. Umgekehrt kann nat�rlich auch schnell �berpr�ft werden, ob die vorher errechneten Werte auch die gew�nschte Sprungantwort liefern. \newline Damit dies so reibungslos wie m�glich klappt, k�nnen die Werte mit den Schieberegler beliebig genau einstellt werden und erm�glichen so eine perfekte Einstellung der Parameter. \newline
Nachdem die Parameter nun eingestellt worden sind, kann der Regler gleich mit �berpr�ft werden, wie dieser mit St�rungen umgeht.\newline
Mithilfe der Speicherfunktion ist es auch au�erdem m�glich, die gefundenen Parameter abzuspeichern und bei Bedarf zu sp�teren Zeitpunkten wieder abzurufen. 

Aus diesem Grund werden wir in unserem zuk�nftigen Elektroniker-Alltag mit Sicherheit immer wieder auf dieses Programm zur�ckgreifen, um einen passenden Regler auszuw�hlen und einzustellen.

%---------------------------------------------------------
\section{Ausblick}
Im Moment berechnet die Software nur ein Blockdiagramm mit einem Regler, einer Strecke und einer St�rung. Die Software kann nun so erweitert werden, dass verschieden Bl�cke in einem separatem Fenster zusammengeschaltet werden k�nnen. Des Weiteren k�nnen so auch verschiedene Blockkomponenten implementiert werden z.B. ein Block, welcher den Logarithmus des Eingangssignals berechnet.\newline
Auf Seiten der Speicher und Lade-F�higkeit der Software kann eine Datenbank erstellt werden, die eine Vielzahl von voreingestellten Parametern bzw. Reglern und Strecken aufweist. So kann nach einer �hnlichen Problemstellung auf Einstellungen zur�ckgegriffen werden, die die grobe Auslegung schon voreingestellt hat. Bei Bedarf kann auf Basis dieser Werte dann Feintuning betrieben werden. \newline
Bei hohen Werten f�llt auf, dass das Programm einige Zeit braucht, da die Anzahl der berechneten Werte sehr hoch wird und die Oberfl�che, solange die Rechnungen nicht abgeschlossen sind, nicht mehr reagiert. Um hier entgegenzuwirken, k�nnten die Berechnungen in einen separaten Prozess ausgelagert werden, sodass zumindest die Oberfl�che nicht komplett einfriert.
