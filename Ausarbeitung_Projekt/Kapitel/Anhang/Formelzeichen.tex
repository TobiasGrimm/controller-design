%\renewcommand\arraystretch{1.5}  % Multiplikationsfaktor f�r die
                                 % Zeilenumbr�che in der Tabelle
\begin{longtable}[t]{p{4cm} p{11.5cm}<{\raggedright}}

\multicolumn{2}{l}{\textbf{Konstante Gr��en}}\\
  $e$                         & Eulersche Zahl, $e \approx 2,71828183$\\
  $\pi$                       & Kreiszahl,    $\pi \approx 3.14159265$\\
  
\multicolumn{2}{l}{\textbf{Formelzeichen}}\\
  $R$                         & Widerstand\\
  $C$                         & Kapazit�t\\
  $L$                         & Induktivit�t\\
  $u$													& Spannung\\
  $i$                         & Stromst�rke\\
  $t$													& Zeit\\
  $f$                         & Frequenz\\
  $n$                         & Drehzahl\\
  $\omega$                    & Kreisfrequenz\\
  $\varphi$                   & Elektrischer Winkel\\
  $M$                         & Drehmoment\\
 
\multicolumn{2}{l}{\textbf{Einheiten}}\\
   V                         & Volt\\
   W           							 & Watt\\
  $\Omega$                   & Ohm\\
	 S                         & Siemens\\
   F                         & Farrad\\
	 H                         & Henry\\
   s                         & Sekunde\\
   m					    					 & Meter\\
   A											   & Ampere\\
   Hz                        & Hertz\\
	 N                         & Newton\\
   B                         & Byte (8 Bit)\\

  
  
\multicolumn{2}{l}{\textbf{Vors�tze}}\\
   n                         &$10^{-9}$\\
   $\mu$                    &$10^{-6}$\\
   m                         &$10^{-3}$\\
   k                         &$10^{3}$\\
   M                         &$10^{6}$\\
\multicolumn{2}{l}{\textbf{Funktionen}}\\
  $\ln$                        & Logarithmus Naturalis\\
  $\sin$                      & Sinus\\
  $\cos$                       & Kosinus\\
  $\tan$                       & Tangens\\
  sig                       & Sigmoid\\
\end{longtable}
